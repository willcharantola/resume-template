% GitHub Repo and Documentation: https://github.com/celiobjunior/resume-template
% Copyright © 2025 Celio B Junior. All rights reserved.
% 
% Licensed under the Apache License, Version 2.0 (the "License");
% you may not use this file except in compliance with the License.
% You may obtain a copy of the License at
%
%     http://www.apache.org/licenses/LICENSE-2.0
%
% This template follows best practices from README.md

% Início do documento LaTeX: define tipo e formato do documento
% a4paper = tamanho A4, 10pt = tamanho base da fonte
\documentclass[a4paper,10pt]{article}

% --- PACOTES ---
\usepackage[utf8]{inputenc}
\usepackage[T1]{fontenc}
\usepackage[portuguese]{babel}
\usepackage{geometry}
\usepackage{parskip}
\usepackage{hyperref}
\usepackage{titlesec}
% Para quebras de linha em URLs longas (não use \href{}, use \url{} para URLs longas)
% \usepackage{xurl}

% --- CONFIGURAÇÃO DO DOCUMENTO ---
% Define margens da página para maximizar espaço de conteúdo
\geometry{top=1.0cm, bottom=1.0cm, left=1.0cm, right=1.0cm}

% Remove números de página e cabeçalhos para visual limpo do currículo
\pagestyle{empty}

% Metadados do PDF - personalize com suas informações
\hypersetup{
    pdftitle={CV Willian Charantola da Costa},
    pdfauthor={Willian Charantola da Costa},
    colorlinks=true,
    linkcolor=black,
    urlcolor=black,
    citecolor=black,
    bookmarksdepth=1 
}

% Desabilita numeração das sections
\setcounter{secnumdepth}{0}

% Formata os cabeçalhos de cada section e coloca uma linha em baixo
\titleformat{\section}
{\Large\bfseries}
{}
{0em}
{}
[\titlerule\vspace{0.5ex}]

% --- INÍCIO DO DOCUMENTO ---
\begin{document}

% --- CABEÇALHO ---
% Substitua com suas informações pessoais
\begin{center}
    {\LARGE \textbf{Willian Charantola da Costa}} 
    \\ [0.1cm]
    Três Lagoas, Mato Grosso do Sul
    \\ [0.1cm]
    {\textbullet}
    Email: \href{mailto:williancharantola14l@gmail.com}{williancharantola14l@gmail.com} 
    {\textbullet}
    Linkedin: \href{https://www.linkedin.com/in/willian-charantola-674b72172/}{linkedin.com/in/willian-charantola-674b72172} 
    \\ [0.1cm]
    {\textbullet}
    Github:\href{https://github.com/willcharantola}{github.com/willcharantola}
\end{center}

% --- SEÇÕES ---


% ------

% !!!!!!!!!!!!!!!!!!!!!!
% Mude esta section para o começo, depois do header e antes
% das suas experiências se você está procurando sua primeira
% vaga ou algum estágio. Do contrário, deixe como está.
% !!!!!!!!!!!!!!!!!!!!!!

\section{Educação}
    % Formação mais recente primeiro
    \subsection*{\texorpdfstring{
            \textbf{Universidade Federal de Mato Grosso do Sul} \hfill Três Lagoas/MS
        }{
            Universidade Federal de Mato Grosso do Sul (UFMS) -- Campus Três Lagoas
        }}
    \textit{Bacharelado em Sistemas de Informação \hfill Cursando - Data de Término (Dez 2026)}






\section{Experiência}
    % Liste as suas experiência de trabalho ou acadêmicas, mais recente primeiro
    \subsection*{\texorpdfstring{
            \textbf{Prefeitura Municipal de Três Lagoas} \hfill Presencial
        }{
            Prefeitura Municipal de Três Lagoas -- Três Lagoas/MS
        }}
    \textit{Analista de Dados/Desenvolvedor \hfill 2022 - 2026}
        \begin{itemize} 
            % Foque em conquistas, não apenas na parte técnica - use o método STAR (Situação, Tarefa, Ação, Resultado)
            \item Desenvolvi uma aplicação web para monitoramento das metas dos instrumentos de planejamento municipais utilizando tecnologias como React, Node e GitHub, resultando em uma redução de 2 semanas no tempo de análise, tratamento e divulgação dos dados.
            \item Desenvolvi uma aplicação web para divulgação dos resultados alcançados das metas dos instrumentos de planejamento municipais utilizando tecnologias como React, Node e GitHub, resultando na melhora da transparência dos resultados alcançados pela administração municipal.
            \item Colaborei com o Comitê de Governança Municipal desenvolvendo um sistema de gestão das respostas do IMG 100 pontos (um modelo de excelência em Gestão e Governança Pública), melhorando o monitoramento e validação das respostas e contribuindo para que o município alcançasse a melhor nota do estado.
        \end{itemize}

        
    \subsection*{\texorpdfstring{
        	\textbf{Projeto Línguas Indígenas MS} \hfill Remoto
        }{
            Universidade Federal de Mato Grosso do Sul -- Remoto
        }}
    \textit{Desenvolvedor Front-end \hfill 2025 - 2026}
        \begin{itemize}
             \item Desenvolvi uma aplicação web para a divulgação de pesquisas sobre línguas indígenas do Mato Grosso do Sul. Utilizando React, Node.js e Figma, entreguei uma interface responsiva e intuitiva, com controle de versão via GitHub.
           \item Link do Projeto: \href{https://linguas-indigenas.github.io/projeto/}{linguas-indigenas.github.io/projeto/}
        \end{itemize}
     

% ------

\section{Habilidades}
    % Mantenha esta seção concisa e use o máximo de palavras chaves
    % que fazem sentido para a vaga que você deseja.
    \begin{itemize}
        \item \textbf{Linguagens de Programação:}  Javascript | Typescript | Java | C | SQL 
        \item \textbf{Desenvolvimento Front-end:}  React | Next.js | HTML | CSS 
        \item \textbf{Desenvolvimento Back-end:}  Node.js | Express.js 
        \item \textbf{Banco de dados:}  PostgreSQL | MySQL 
        \item \textbf{Versionamento e Repositório:}  Git | GitHub  
        \item \textbf{Idiomas:} Inglês (Intermediário).
    \end{itemize}

\section{Projetos de Destaque} 
    
    % Projeto 1: Programa de Metas
    \subsection*{\texorpdfstring{
            \textbf{Programa de Metas} \hfill Três Lagoas, MS [cite: 2, 11]
        }{
            Programa de Metas -- Três Lagoas, MS
        }}
        \begin{itemize}
            \item Desenvolvimento de uma aplicação web responsiva para a transparência e divulgação dos resultados dos instrumentos de planejamento municipais utilizando React e Vite[cite: 16].
            \item Implementação de gráficos dinâmicos para visualização de dados complexos, tornando a apresentação das metas mais intuitiva para o cidadão[cite: 16].
            \item Link do Projeto: \href{https://planejamentotl.github.io/programa_metas/}{https://planejamentotl.github.io/programa_metas/} 
        \end{itemize}

    % Projeto 2: Línguas Indígenas MS
    \subsection*{\texorpdfstring{
            \textbf{Projeto Línguas Indígenas MS} \hfill Remoto [cite: 20]
        }{
            Projeto Línguas Indígenas MS -- Remoto
        }}
        \begin{itemize}
            \item Criação de uma interface web moderna com React e Vite dedicada à valorização e divulgação de estudos sobre as línguas indígenas do Mato Grosso do Sul[cite: 21, 22].
            \item Desenvolvimento de componentes reutilizáveis e design responsivo planejado no Figma para garantir acessibilidade em diversos dispositivos[cite: 22].
            \item Link do Projeto:   \href{https://linguas-indigenas.github.io/projeto/}{linguas-indigenas.github.io/projeto/}
        \end{itemize}

% ------
\end{document}


